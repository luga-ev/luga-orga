\documentclass[10pt,a4paper,ngerman,landscape]{scrartcl}
\usepackage[utf8]{inputenc}%latin1
%\usepackage{ngerman}
%\usepackage{pslatex}
%\usepackage{pdflscape}
%\renewcommand{\familydefault}{\sfdefault}
% Papierformat auf DIN-A4
%\usepackage{setspace}
\usepackage[ngerman]{babel}
\usepackage[protrusion=true,expansion=true]{microtype}
\usepackage{lmodern}
\usepackage{amssymb}
\usepackage{multicol}
\usepackage{multirow}
\usepackage[paper=a4paper,left=11mm,right=11mm,top=14mm,bottom=14mm]{geometry} 
% Made changes based on suggestions from Gene Cooperman. <gene at ccs.neu.edu>

% Turn off header and footer
\pagestyle{empty}
 
% Redefine section commands to use less space
\makeatletter
%\renewcommand{\section}{\@startsection{section}{1}{0mm}%
%                                {-1ex plus -.5ex minus -.2ex}%
%                                {0.5ex plus .2ex}%x
%                                {\normalfont\large\bfseries}}
%\renewcommand{\subsection}{\@startsection{subsection}{2}{0mm}%
%                                {-1explus -.5ex minus -.2ex}%
%                                {0.5ex plus .2ex}%
%                                {\normalfont\normalsize\bfseries}}
%\renewcommand{\subsubsection}{\@startsection{subsubsection}{3}{0mm}%
%                                {-1ex plus -.5ex minus -.2ex}%
%                                {1ex plus .2ex}%
%                                {\normalfont\small\bfseries}}
\makeatother

% Don't print section numbers
\setcounter{secnumdepth}{0}


\setlength{\parindent}{0pt}
\setlength{\parskip}{0pt plus 0.5ex}

\let\origdescription\description
\renewenvironment{description}{
  \setlength{\leftmargini}{0em}
  \origdescription
  \setlength{\itemindent}{0em}
  \setlength{\itemsep}{1.2em}
  \setlength{\labelsep}{\textwidth}
}
{\endlist}

\newcommand{\vorschub}{\mbox{}\\[-0.5em]}


% -----------------------------------------------------------------------

\begin{document}

%\begin{landscape}
\begin{center}
     {\huge \textbf{\sffamily 13. Augsburger Linux-Infotag}}\\[2ex]
     organisiert von der Linux User Group Augsburg (LUGA) e.\,V. zusammen
     mit der Hochschule Augsburg
\end{center}

\vspace{2ex}

\section{Übersicht über das Programm}
%\footnote{
{
\begin{table}[h!]
   \small\sffamily
        \begin{center}
        \begin{tabular}{|c||l|l|l||l|} \hline
            \bfseries{09:50} & \multicolumn{4}{c|}{\rule[-3mm]{0mm}{8mm}{\bfseries{Begrüßung}}}  \\ \hline
            \bfseries{10:00} &
            \multicolumn{4}{c|}{\rule[-3mm]{0mm}{8mm}{\bfseries Keynote: Patricia Jung}}\\ \hline
                             & \multicolumn{4}{c|}{\rule[-3mm]{0mm}{8mm}{\bfseries{Vortragsraum}}} \\ \cline{2-5} 
                             & \rule[-3mm]{0mm}{8mm}\bfseries{A (M1.02)}
                             & \bfseries{B (J2.18)} & \bfseries{C (J3.19) } & \bfseries{D (M2.03)} (Workshops) \\ \hline  \hline  
            \bfseries{11:00} & \rule[-3mm]{0mm}{8mm} $\bigstar$ Open Source in Unternehmen &
                                                     $\bigstar$ Kollaboration mit Etherpad &
                                                     $\phantom{\bigstar}$
                                                     Windows- und Linux-Passwörter zurücksetzen &
                                                     \multirow{3}{*}{Ruby on Rails} \\
                                                     \cline{1-4}
            \bfseries{12:00} & \rule[-3mm]{0mm}{8mm} $\bigstar$ Geotagging. Fotos mit Geoinformationen verknüpfen &
                                                     $\phantom{\bigstar}$ OpenStack &
                                                     $\bigstar$ Reprap-3D-Drucker &
                                                     \\ \hline
            \bfseries{13:00} & \rule[-3mm]{0mm}{8mm} $\bigstar$ Linux für alle oder doch nicht? &
                                                     $\bigstar$ ownCloud -- meine Daten gehören mir! &
                                                     $\phantom{\bigstar}$ Warum Django? &
                                                     \\ \hline
%           \bfseries{13:00} & \multicolumn{4}{c|}{\rule[-3mm]{0mm}{8mm}{\bfseries{Mittagspause}} }\\ \hline
            \bfseries{14:00} & \rule[-3mm]{0mm}{8mm} $\bigstar$ Quo vadis, IT-Sicherheit? &
                                                     $\bigstar$ The Document Foundation &
                                                     $\phantom{\bigstar}$ Entwicklungen beim Linux-Kernel &
                                                     \multirow{3}{*}{Linux im Musikstudio} \\ \cline{1-4}
            \bfseries{15:00} & \rule[-3mm]{0mm}{8mm} $\bigstar$ Bitcoin -- Open Sourcing Money &
                                                     $\phantom{\bigstar}$ Mehrere Terminals in einem mit screen &
                                                     $\bigstar$ Unabhängige Datenverwaltung mit ownCloud &
                                                     \\ \hline 
            \bfseries{16:00} & \rule[-3mm]{0mm}{8mm} $\bigstar$ HAMNET &
                                                    $\bigstar$ Red Hat Certified Systems Administrator &
                                                    $\phantom{\bigstar}$ Wireshark ohne Netzwerk &
                                                    \\ \hline 
            \bfseries{17:00} & \multicolumn{4}{c|}{\rule[-3mm]{0mm}{8mm}{\bfseries{Verlosung}}} \\ \hline
            \bfseries{17:30} & \multicolumn{4}{c|}{\rule[-3mm]{0mm}{8mm}{\bfseries{Ende der Veranstaltung}}} \\ \hline
          \end{tabular}
        \end{center}
\end{table} 
}
\vspace{-1.0em}
Mit $\bigstar$ markierte Vorträge benötigen kein tiefergehendes Vorwissen.

Kurzfristige Programmänderungen können nicht ausgeschlossen werden,
bitte beachten Sie die Programmhinweise in der Aula.

Den ganzen Tag über hat
die Mensa der Hochschule geöffnet. Dort kann bar bezahlt werden.

\section{Keynote: 10:00 -- 10:45 Uhr}
Eröffnungsvortrag von Patricia Jung, ehemalige stellvertretende Chefredakteurin
von Linux-User.

\newpage

\setlength{\columnsep}{10.5mm}
\begin{multicols}{3}

%\end{landscape}

%\raggedright

\subsection{11:00 -- 11:45 Uhr}
%\item[{\parbox[t]{\linewidth}{PostgreSQL – Neues und Besonderes der führenden
%Open-Source-Datenbank}}]\vorschub\\
%\item[ownCloud – Your Cloud, Your Data, Your Way]\vorschub
\begin{description}
\item[Open Source in Unternehmen]\vorschub
\textsl{Ulrich Habel.}
Der Einsatz von Open-Source-Software und Linux als Betriebssystem stellt jedes
Unternehmen vor größere Hürden. Welche Punkte sind zu beachten -- wie kann das
Unternehmen selber Teil eines Open-Source-Projekts werden und welche
Auswirkungen hat das auf die Mitarbeiter?

Der Vortrag kommt direkt aus der Praxis und behandelt den Einsatz von
Open-Source-Software in einem mittelständischen Betrieb. Er weist auf die
offensichtlichen und verdeckten Stolpersteine hin und stellt einen Leitfaden
vor, der unmittelbar anwendbar ist. Selbstverständlich wird auf
betriebswirtschaftliche Aspekte eingegangen.

Dieser Vortrag eignet sich gleichermaßen für Einsteiger, die in dem Bereich
Administration arbeiten möchten, für Unternehmen die Kochrezepte suchen und für
alle Interessierten, die Zusammenhänge verstehen möchten.



\item[Kollaboration mit Etherpad]\vorschub
\textsl{David Krcek.}
Durch Vernetzung über das Internet arbeiten immer mehr Menschen an verschieden
Orten an gleichen Projekten. Durch diese Entwicklung nimmt der Bedarf an der
gemeinsamen Bearbeitung von Texten in Echtzeit immer mehr zu.

Etherpad ist das Werkzeug für die gemeinsame Bearbeitung von Texten in Echtzeit.
Durch Plattformunabhängigkeit, Verschlüsselung und Erweiterungsmöglichkeiten
sind Etherpad kaum Grenzen gesetzt. In diesem Vortrag werden der Aufbau und die
Leistungsfähigkeit skizziert sowie die einfache Bedienung live
demonstriert.


\item[{\parbox[t]{\linewidth}{Windows- und Linux-Passwörter mit Minilinux zurücksetzen}}]\vorschub\\
\textsl{Oliver Rath.} Bei Computer-Anwendern kommt es gelegentlich vor, dass ein Passwort
vergessen wird. Unter Linux ist es relativ einfach, offline Passwörter
zurückzusetzen. Bei Windows hat man bei den Werkzeugen oft das Nachsehen.
Doch dank ntpasswd und der Möglichkeit, ein Linux offline zu booten, ist
hier oft Abhilfe möglich. Wir zeigen die generische Anwendung der Werkzeuge
und die möglichen Fallstricke bei Linux und Windows.

Wenn noch Zeit bleibt, erzeugen wir uns noch unser eigenes, kleines
Mini-Linux für diesen Zweck.

\item[{\parbox[t]{\linewidth}{Workshop: Schnelle Anwendungsentwicklung mit \mbox{Ruby} on Rails}}]\vorschub\\
\textsl{Thomas Eisenbarth.}
In diesem Vortrag wird die Programmiersprache Ruby und das Web-Framework
Rails vorgestellt. Anhand einer kleinen Demo-Anwendung wird gezeigt,
warum sich Ruby on Rails hervorragend eignet, um schnell professionelle
Web-Anwendungen zu erstellen. Darüber hinaus wird -- ebenfalls als
Demonstration -- der Hintergrund von Testgetriebener Softwareentwicklung
(TDD) gezeigt.

\end{description}

\subsection{12:00 -- 12:45 Uhr}
\begin{description}
\item[Geotagging. Fotos mit Geoinformationen verknüpfen]\vorschub
\textsl{Frank Hofmann.}
Nach jeder Reise gleicht das Sortieren der Mitbringsel und Zuordnen der Fotos
häufig einem anspruchsvollen Puzzlespiel. Auch unsere Mitmenschen danken es
uns, wenn sie später nicht nur endlose Pixelberge gezeigt bekommen, sondern die
Fotos auf einer Landkarte mitverfolgen und geographisch einsortieren können. Im
Mittelpunkt stehen Linux-Bordmittel und frei verfügbare Dienste wie
beispielsweise OpenStreetMap.



\item[{\parbox[t]{\linewidth}{OpenStack -- automatisiertes Bereitstellen von Test-, Entwicklungs-
\& Produktivinstanzen in einer Private Cloud}}]\vorschub\\
\textsl{Ralph Dehner.}
OpenStack ist aktuell die Cloud-Lösung weltweit, die das größte Aufsehen als
Private-Cloud-Lösung erzeugt. B1 Systems zeigt auf, wie innerhalb der Lösung
unterschiedliche virtuelle Instanzen über die Selbstprovisionierung
bereitgestellt werden können.

\item[{\parbox[t]{\linewidth}{Reprap-3D-Drucker -- Open Hardware und Open Software in
idealer Kombination}}]\vorschub\\
\textsl{Stefan Krister.}
Das Reprap-Projekt hat als Ziel, eine sich selbst replizierende Maschine zu
entwickeln. Seit 2005 arbeitet Adrian Bowyer an der Verwirklichung. Der Vortrag
zeigt die heutige Situation und führt in das Konzept des 3D-Drucks ein. Es
werden die Komponenten (mechanisch, elektrisch und elektronisch) vorgestellt
und der Software-Arbeitsablauf von der Konstruktion bis zum gedruckten Objekt
gezeigt.

Verwendete Hardware: Mendel90 Drucker, Raspberry Pi, Arduino;
verwendete Software: OpenSCAD, Slic3r, Octoprint/Octopi




\end{description}

\subsection{13:00 -- 13:45 Uhr}
\begin{description}

\item[Linux für alle oder doch nicht?]\vorschub
\textsl{Richard Albrecht.}
Linux ist eine Erfolgsgeschichte, die vor 20 Jahren nicht absehbar war. Wenn
man Linuxtage besucht, ist Linux überall, jeder hat es, jeder kennt es, alles
ist einfach, flexibel und vielfältig. Es erscheint aus der Linuxtagperspektive
unvorstellbar, dass es Leute gibt, die noch nie etwas von Linux gehört haben,
für die Linux als sehr kompliziert erscheint.

Schaut man sich die PCs dieser Leute zu Hause an, dann ist Linux praktisch
nicht im Einsatz. Der Durchbruch von Linux auf dem Desktop ist bei ihnen nicht
angekommen, trotz aller Begeisterung auf Linuxtagen und an
LUG-Stammtischen.

Im Vortrag möchte ich über meine Erfahrungen mit ganz normalen PC Nutzern beim
Umstieg nach Linux berichten. Sie erfahren Linux als pflegeleicht und stabil
und die eigenen Computerkenntnisse werden besser.

Am Beispiel einer Schule in Kroatien, die nur PCs mit Microsoft Systemen
einsetzte, möchte ich zeigen, was sich entwickeln kann, wenn man Pro-Linux
Stimmen, die immer vorhanden sind, aufnimmt und die Arbeit mit Linux
aufbaut.




\item[ownCloud -- meine Daten gehören mir!]\vorschub
\textsl{Björn Schießle.}
In einer Zeit, in der sich Cloud Computing immer größerer Beliebtheit erfreut,
ist freie Software alleine nicht mehr ausreichend um die Kontrolle über sein
digitales Leben zu behalten. Wir wollen von überall und jederzeit mit einer
Vielzahl von Geräten auf unsere persönlichsten Daten zugreifen.

Um dies zu
ermöglichen, speichern wir unser Daten nicht mehr auf dem heimischen Computer,
sondern auf irgendwelchen Servern die sich unserer Kontrolle entziehen. Wer
kann alles darauf zugreifen? Wer bestimmt über den Zugang zu den Daten? Wo
genau werden die Daten überhaupt gespeichert?

Gerade heute, ein Jahr nach den
Enthüllungen von Edward Snowden, sind diese Fragen aktueller denn je. own\-Cloud
ermöglicht es, die Kontrolle über die eigenen Daten zurück zu erlangen, egal ob
Dokumente, Bilder, Kontakte, Kalender, Mediastreaming oder vielem mehr. Mit
ownCloud kontrolliert man nicht nur die Software, sondern auch wo die Daten
liegen, wer darauf Zugriff hat und was damit passiert.


\item[Warum Django?]\vorschub
\textsl{Jürgen Schackmann.}
Django ist ein Open-Source-Web-Framework für die Programmiersprache Python.
Mit Django lassen sich Webseiten und Web-Anwendungen schnell und mit
wenig Code implementieren. Im Vortrag werden folgende Themen behandelt:
Grundlegende Django-Konzepte (Projektstruktur, MVC, ORM, Templating etc.),
typischer Entwicklungsprozess in einem Django-Projekt,
Django-Community und Ökosystem, Sicherheit und Django für Enterprise-Anforderungen.
\end{description}

\subsection{14:00 -- 14:45 Uhr}

\begin{description}
\item[Quo vadis, IT-Sicherheit?]\vorschub
\textsl{Thomas Eisenbarth.}
Snowden und die NSA $\bullet$ Deutsche Telekom und das deutsche Internet $\bullet$ GMX \& Co.
mit E-MAIL MADE IN GERMANY $\bullet$ RSA und der Zufallsgenerator $\bullet$ SSL und der grüne
Balken $\bullet$ Lavabit und Kryptographie. sigh!

Resümieren wir das vergangene Jahr 2013, muss einem unter
IT-Sicherheitsaspekten fast schon schlecht werden ob der permanenten
Horror-Nachrichten, die uns erreicht haben. Dieser Vortrag wird (pointiert)
beleuchten, was alles schief gegangen ist, schief gehen wird und was man
dagegen tun kann: Wem oder was kann man noch trauen und wie kann man sich
schützen? Quo vadis, IT-Sicherheit?




\item[{\parbox[t]{\linewidth}{The Document Foundation -- ein Blick hinter die Kulissen}}]\vorschub\\
\textsl{Florian Effenberger.}
The Document Foundation ist die gemeinnützige Stiftung hinter LibreOffice, der
freien Office-Suite. Sie stellt nicht nur einen rechtlichen Rahmen für das
Projekt bereit, sondern verwaltet zudem auch Rechtsgüter und Spenden, um die
Fortentwicklung der Software sowie der Community sicherzustellen. Der Aufbau
der Document Foundation ist dabei einzigartig -- und durch die Wahl einer
deutschen Stiftung hat sich die Community für eine Organisationsform
entschieden, die weltweit als stark, stabil und dauerhaft wertgeschätzt wird.
Ihre Satzung stellt die Unabhängigkeit von einem einzelnen Unternehmen sicher
und hebt gleichzeitig Transparenz, Offenheit und Meritokratie als zentrale
Werte hervor.

In dem Vortrag gibt Florian Effenberger, einer der Ini\-ti\-a\-to\-ren der TDF, einen
Einblick in den Aufbau der Stiftung, darüber, welche Ziele sie in den letzten
25 Monaten erreicht hat, wie ein Projekt dieser Größenordnung koordiniert wird,
und welche Pläne und Visionen das Projekt für die Zukunft hat.


\item[Aktuelle Entwicklungen beim Linux-Kernel]\vorschub
\textsl{Thorsten Leemhuis.}
Der Vortrag gibt einen Überblick über die jüngsten Verbesserungen beim
Linux-Kernel, denn die sind oft auch für Allerwelt-PCs oder Server von Belang;
mit Distributionen wie Ubuntu 14.04 erreichen die Verbesserungen der neuesten
Kernel in Kürze auch eine breite Anwenderschar.

Der Vortrag geht auch auf einige Neuerungen bei Kernel-naher Software ein --
etwa den Open-Source-3D-Grafiktreibern. Angerissen werden auch einige noch in
Vorbereitung befindliche Änderungen, der Entwicklungsprozess sowie andere
Aspekte rund um den Kernel, die für die kurz- und langfristige Entwicklung von
Linux und Linux-Distributionen wichtig sind.

Zielpublikum des Vortrags sind technisch interessierte Linux-Nutzer.




\item[Workshop: Linux im Musikstudio]\vorschub
\textsl{Franz Tea.}
Für Musikschaffende im Hobby- oder professionellen Bereich kann Linux eine gute
Basis für ihre Arbeit darstellen. Ein Echtzeit-Linux wie etwa Ubuntu Studio mit
den entsprechenden Programmen deckt fast den ganzen Bereich ab: von der
Musikerzeugung über die Aufnahme bis zur Abmischung. Die Simulation von
Synthesizern und anderen Geräten, der Einsatz von Effekten, ausgereifte
Programme zum Abmischen, ein Festplatten-Recorder, die Anbindung vorhandener
Geräte wie zum Beispiel Keyboards -- alles ist vorhanden. Durch die
ausgeklügelte Architektur der Schnittstellen in Linux ist der Baukasten an
Geräten einfach virtuell zusammenzustöpseln. In dem Workshop werden wir
gemeinsam versuchen, ein Musikstück von der Idee bis zur Fertigstellung zu
bringen.
\end{description}


\subsection{15:00 -- 15:45 Uhr}

\begin{description}
\item[Bitcoin -- Open Sourcing Money]\vorschub
\textsl{Levin Keller.}
Bitcoin ist ein länderübergreifendes Zahlungssystem in Form von virtuellem Geld. Die Über\-tra\-gung der Beträge erfolgt direkt von Teilnehmer zu Teilnehmer (Peer-to-Peer). Dadurch werden die beim herkömmlichen Bankverkehr üblichen Zwischenschritte umgangen.

Im Jahr 2013 hat die Bitcoin erstmalig breite
Medienaufmerksamkeit erfahren. Dabei werden vor allem der rasante
Kursanstieg und Warnungen durch Regierungsvertreter thematisiert. Eine
genaue Erklärung von Bitcoin und der Potentiale der Technologie bleibt
leider meistens aus.
\vfill
\columnbreak

Levin Keller wird in seinem Vortrag zunächst die
Bit\-coin zugrundeliegende Technologie erläutern, um dann einige
Möglichkeiten zu skizzieren, die sich in Zukunft durch die Verwendung von
Bitcoin ermöglichen. Hierzu zählen u.\,A. zensurresistente Geldflüsse an
Bürgerrechtsorganisationen, Crowdfunding und Spenden für
Open-Source-Software und Weiterentwicklungen von Bitcoin wie Namecoin,
Ripple oder Ethereum.

Im Anschluss steht Levin Keller für Fragen oder Hilfe bei der
Installation der Bitcoin-App auf dem eigenen Handy zur Verfügung.




\item[Mehrere Terminals in einem mit GNU Screen]\vorschub
\textsl{Axel Beckert.}
GNU Screen erlaubt einem, in einem Text-Terminal mehrere Kommandozeilenshells
und andere Anwendungen gleichzeitig laufen zu lassen. Es ermöglicht einem
außerdem Text-Modus-Anwendungen oder Shells weiterlaufen zu lassen, auch wenn
man sich (etwa per SSH) ausgeloggt hat. Man kann auch später vom selben oder
von einem anderen Rechner aus per SSH sich wieder mit den laufenden Anwendungen
verbinden.

Der Vortrag geht auch kurz auf die Screen-Alternative Tmux sowie auf Byobu ein,
welches auf Tmux und Screen aufbaut.

\item[{\parbox[t]{\linewidth}{Sichere und unabhängige Datenverwaltung mit ownCloud}}]\vorschub\\
\textsl{Bernd Müller.}
Speicherplatz in der Cloud wird immer beliebter, nicht nur für Unternehmen
und virtuelle Systeme, sondern auch für Endanwender bzw. Nutzerdaten. Vorteile
wie Sicherheitskopien, Daten teilen, Kontakte, Kalender, Browserdaten
synchronisieren usw. werden von unterschiedlichen Unternehmen und
Community-Projekten angeboten.

Dieser Vortrag zeigt, wie mit Hilfe der freien Software ownCloud diese Dienste
auf einfache Art und Weise selbst betrieben werden können. Für viele
Unternehmen und Anwender ist eine wichtige Voraussetzung, dass die Daten nicht
in anderen Ländern oder fremden Rechenzentren liegen. Zusätzlich bietet
ownCloud die Möglichkeit, die Daten verschlüsselt abzulegen.

Für ownCloud sprechen auch die zahlreichen unterstützten Zugriffsmöglichkeiten.
Es existieren Programme bzw. Apps für Linux, Microsoft, Mac OS, Android und
iOS. Ganz gleich, ob es um die gemeinsame Arbeit an einem Dokument,
Terminkoordination oder schlicht die gemeinsame Nutzung von Daten geht: In
diesem Vortrag wird gezeigt, wie von verschiedenen Geräten und Plattformen auf
Daten zugegriffen und diese geteilt werden können. Auch im Hinblick auf
Sicherheitsaspekte und den Schutz der eigenen Privatsphäre erfährt der Zuhörer,
welche individuellen Möglichkeiten ihm und seinem Unternehmen dank Community-
und Corporate-Sparten bei ownCloud und Partnern zur Verfügung stehen.
\end{description}

\subsection{16:00 -- 16:45 Uhr}

\begin{description}
\item[HAMNET]\vorschub
\textsl{Olaf Henne.}
HAMNET ist ein Funknetz von Funkamateuren für Funkamateure. Es reicht von dem
Mittelmeer bis zur Nordsee und die gesamte Ausrüstung ist privat finanziert.
Damit ist klar, dass nach Möglichkeit auf freie Software und günstige Hardware
zurückgegriffen wird. In diesem Funknetz werden unterschiedliche Dienste
bereitgestellt, zum Beispiel Suchmaschinen, Messenger, Sprachkonferenzen,
etc.

Funkamateure dürfen experimentieren. Sie dürfen also diese Dienste an die
Besonderheiten eines Funknetzes anpassen. Sie können neue Lösungen für die
unterschiedlichsten Probleme (zum Beispiel unterschiedliche Dämpfung bei
Regen/Sonne oder Sommer/Winter) im Frequenzbereich um 2,3 bzw. 5,7 GHz
entwickeln und unter realen Bedingungen ausprobieren. Open-Source-Software ist
aus vielen Gründen für diese Aufgaben optimal.

In dem Vortrag werden das HAMNET und die dort vorhandenen Dienste, einige
Geräte für das HAMNET und manche Probleme vorgestellt, um gegebenenfalls
Synergien zu anderen Projekten zu finden.

\columnbreak
\item[{\parbox[t]{\linewidth}{Buchvorstellung: Red Hat Certified Systems Administrator}}]\vorschub\\
\textsl{Dieter Thalmayr.}
Im Februar erschien die erste Auflage des Buchs "`RHCSA"' beim
opensourcepress-Verlag. Mein drittes Buch ist kein reines Linux-Einsteigerbuch,
sondern es zielt darauf ab, den Leser auf die Prüfung zum "`Red Hat Certified
Systems Administrator"' vorzubereiten. Freilich hilft es auch dabei,
Administratoren, die andere Linuxe kennen, den Ein- und Umstieg auf
Redhat-artige Distributionen zu erleichtern. Und es hilft Einsteigern, Linux zu
lernen.

Der "`RHCSA"' ist der Einsteigergrad aller Red-Hat-Zertifizierungen. Durch
den Wegfall von Konkurrenz ist die Red-Hat-Zertifizierung in Deutschland im
Moment die einzige nennenswerte Industrie-Zertifizierung für große Oracle- und
SAP-Umfelder geworden. Worum es bei der Prüfung geht, wie man sie ablegt, und
wie sich das Leben mit einem Linux-Monopolisten anfühlt, darum geht es in
diesem Vortrag.

\item[Wireshark ohne Netzwerk]\vorschub
\textsl{Martin Kaiser.}
Fast jeder von Euch hat Wireshark schon genutzt, um Netzwerktraffic
mitzuschneiden.

Die vielfältigen Möglichkeiten der Darstellung und Filterung sind auch für
andere Protokolldaten interessant, die zum Beispiel von Messgeräten oder selbstgebauter
Hardware stammen. Dieser Vortrag zeigt anhand von Beispielen, wie man solche
Daten für Wireshark lesbar macht und welche Möglichkeiten Wireshark bietet,
diese Daten zu analysieren.

Für den Fall, dass Wireshark Euer Protokoll noch nicht kennt, gibt es noch eine
kurze Einführung, wie man ein neues Protokoll implementiert.
\vfill

\end{description}
\raggedleft\rule{0.7\linewidth}{0.25pt}
\scriptsize

\raggedleft\texttt{http://www.luga.de/LIT-2014/}

\end{multicols}

\end{document}
